\documentclass{article}
\usepackage[utf8]{inputenc}
\usepackage{amssymb, amsmath,hyperref,verbatim,listings,graphicx,subfigure,fullpage}

\begin{document}

\title{UML computer project 1}
\author{
Juha-Antti Isojärvi\\
013455341 \\
Department of Mathematics and Statistics\\
Master student
\and
Mikko Sysikaski\\
013573016\\
Department of Computer Science\\
Master student}
\date{}
\maketitle

\section{Exercise set 1}
\subsection{Exercise 1}
First we were given a plot of twodimensional data, and were asked to
create similar data artificially. The distribution of the data looked
similar to the twodimensional gaussian or multinormal
distributions. Therefore we decided to generate points from a
gaussian distribution. 

A random vector $X = (X_1, X_2, \dots, X_n)$ is standard normally
distributed, denoted $X \sim N_n(0,I)$, if its components $X_i$ are independent and normally
distributed with zero mean and unit variance. It is true, that the
mean vector $E(X) = 0$ and covariance matrix $Cov(X) = I_n$.

Now if a random vector 
\[
X = AU + \mu
\] 
is defined for some $U \sim N_n(0,I)$, and fixed $A \in \mathbb{R}^{m \times n}$
and $\mu \in \mathbb{R}^n$, then it is true that $E(X) = \mu$ and
$Cov(X) = AA^T$.

A random vector $X$ is multinormally
distributed, if it has the same
distribution as the random vector
\[
X = AU + \mu.
\]
Then it has mean $E(X) = \mu$ and covariance $Cov(X) = AA^T$. Denote
$Cov(X) \dot{=} \Sigma$. The multinormal distribution is denoted $X
\sim N_n(\mu,\Sigma)$.

It also holds, that if $X \sim N_n(\mu,\Sigma)$, and $B \in
\mathbb{R}^{m \times n}$ and $b \in \mathbb{R}^m$ are fixed, then 
\[
BX + b \sim N_m(B\mu + b, B \Sigma B^T).
\]

Now, one way of simulating gaussian twodimensional data, is to first
simulate twodimensional standard normally distributed, i.e. white, data, and then
affinely transform this data by multiplying with some matrix $A \in
\mathbb{R}^{2\times 2}$. The affinely transformed data has then a
gaussian distribution with zero mean and covariance $AA^T$.

So, we generated two random samples from a normal distribution and put these
together to form twodimensional white data. If one would plot this
data, one would see a spherical point cloud centered at zero. Then we
dilated this cloud in the $y$- and $x$-axis direction with multiplying by a dilation matrix $D$ and then rotated
the dilated data cloud with multiplying by a rotation matrix
$R_\theta$. The result was a sample from the 
distribution $N_2(0,AA^T)$, where $A \dot{=} R_\theta D$.

The resulting data cloud is centered at zero, and has elliptical
shape. The main axis of this data cloud has angle $\theta$ measured
form the $x$-axis. Our resulting four generated artificial data can be
seen plotted in Figure~\ref{fig:scatter}. 

This was a simple method for generating artificial twodimensional
gaussian data, with easy control of the direction and elongation of
the data cloud. Of course there are other methods as well. In the
$MASS$-package of $R$ there is a function mvrnorm, which takes as
inputs the mean vector and the covariance matrix and generates
gaussian data. The method implemented is the so called
Stetson-Harrison method, and we think it may be just what we used. We
didn't feel the need to get further into the details of this
method. Manipulation of the direction and the elongation of the data
cloud is always achieved by manipulation of the covariance matrix, as
in the method we described.

Another way of manipulating the direction and elongation of the data
cloud is by 'inverse' eigenvalue decomposition of the covariance
matrix. About this method more in section \ref{sec:subsection4}.
\subsection{Exercise 2}
The principal components of the first and the third point set are shown in Figure~\ref{fig:pcadir}.
\subsection{Exercise 3}
The Figure~\ref{fig:histo} contains the histograms obtained by projecting the points of the first point set on its principal components.
\subsection{Exercise 4}\label{sec:subsection4}
\subsection{Exercise 5}

\newcommand{\sscale}{0.5}
\begin{figure}\centering
	\includegraphics[scale=\sscale]{scatter1}
	\includegraphics[scale=\sscale]{scatter2}

	\includegraphics[scale=\sscale]{scatter3}
	\includegraphics[scale=\sscale]{scatter4}
	\caption{The scatter plots of the data generated in task 1.1.} \label{fig:scatter}
\end{figure}
\begin{figure} \centering
	\includegraphics[scale=\sscale]{pcadir1}
	\includegraphics[scale=\sscale]{pcadir3}
	\caption{Principal components of the first and the third point set. The first PC is the bolder line.} \label{fig:pcadir}
\end{figure}
\begin{figure} \centering
	\includegraphics[scale=\sscale]{histo1-1}
	\includegraphics[scale=\sscale]{histo1-2}
	\caption{Histograms of the 1-dimensional data obtained by projecting the points of the first point set on its principal components.} \label{fig:histo}
\end{figure}
\begin{figure} \centering
	\includegraphics[scale=\sscale]{vscatter}
	\caption{} \label{fig:vscatter}
\end{figure}
\begin{figure} \centering
	\includegraphics[scale=\sscale]{proj}
	\caption{} \label{fig:proj}
\end{figure}


\clearpage
\section{Exercise 2}
\newcommand{\X}{\ensuremath{\mathbf{X}}}
\subsection{}
The task was to do primary component analysis on the matrix
$$ X =
\begin{pmatrix}
	5 & 3 & 0 & 1 & -1 & -3 & 5 & 0 & -4 & -4 \\
	-2 & -1 & 0 & 0 & 1 & 4 & -3 & 1 & 5 & 3 \\
	0 & 1 & 4 & -1 & 0 & 5 & 5 & -5 & -3 & -3 \\
	0 & 2 & 3 & 0 & -1 & 3 & 3 & -7 & -2 & 0 \\
	3 & 4 & -2 & 1 & 3 & -3 & -3 & 2 & 0 & 0
\end{pmatrix}.
$$

The Figure~\ref{fig:42} displays the data and the original variables projected to the first two principal components.
\subsection{}
The amount of variance explained as a function of the number of principal components used is displayed in Figure~\ref{fig:varamount}. It can be seen that the projection to the first two components in Figure~\ref{fig:42} convey about 90.6\% of the information of the data.
\subsection{}
\subsection{}
The quartimax applied to the first two principal components of \X. Figure~\ref{fig:qmax} shows the projections of the variables on the rotated components. Note that the figure is the same as Figure~\ref{fig:42}, only rotated. The rotation doesn't change the subspace spanned by the principal components, so they explain the same amount of variance before and after rotation.

\begin{figure}\centering
	\includegraphics[scale=\sscale]{fig42}
	\caption{Reproduction of Figure~4.2 on the lecture notes. The blue points are the data points projected to the first two principal components. The red lines are the projections of the original coordinate axes.}\label{fig:42}
\end{figure}
\begin{figure}\centering
	\includegraphics[scale=\sscale]{varamount}
	\caption{The proprotion of the variance explained by using only some of the principal components.}\label{fig:varamount}
\end{figure}
\begin{figure}\centering
	\includegraphics[scale=\sscale]{qmax}
	\caption{The projection to principal components after rotating them using the quartimax algorithm to have the original variables as close to the new coordinate axes as possible.}\label{fig:qmax}
\end{figure}


\clearpage
\section{Exercise 3}
\subsection{}\label{sec:31}
The data was preprocessed by first looking at each digit separately.
For each digit we removed the average of the values and normalized then to unit norm.
Finally we again took all the images together and centered each variable to have mean 0.
The first 20 of the digits before and after preprocessing are displayed in Figure~\ref{fig:digits}.
\begin{figure}\centering
	\includegraphics[scale=0.4]{digits}
	\includegraphics[scale=0.4]{digitspre}
	\caption{Visualization of the first 20 digits of the data before and after preprocessing.
	The preprocessing centerss and normalizes each digit and centers each variable.}\label{fig:digits}
\end{figure}

\subsection{}
The variances of the digit data explained by the number of principal components are displayed in Figure~\ref{fig:pcavar}.
The same figure displays visualizations of the actual principal components.
Because the principal components are vectors of the same length as each digit vector, they can be visualized the samy way as the digits.
As can be expected, the visualization shows that the principal components have non-zero values mostly near the center of the images, where the digits differ a lot from each other.
\begin{figure}\centering
	\includegraphics[scale=0.6]{pcavar}
	\includegraphics[scale=0.4]{pcanums}
	\caption{Left: the proportion of variance explained by the number of principal components.
	Right: visualization of the principal components.
The colors are scaled such that black and white parts of the images represend negative and positive valus in the PCs while grey pixels represent zero-values.}\label{fig:pcavar}
\end{figure}

\subsection{}
Let $Y$ be the $784\times 1000$ matrix obtained from the original matrix $X$ by preprocessing it as described in Section~\ref{sec:31}.
Denote by $P$ the $784\times 784$ matrix containing the principal components of $Y$ as its columns, and by $P^i$ its first $i$ columns.
We can reduce the dimension of $Y$ by multiplying it with $P$: $R^i=Y^TP^i$.
We can then reproduce an approximation to the original matrix by multiplying again by the transpose of $P^i$: $$(Y^i)^T=R^i(P^i)^T=Y^TP^i(P^i)^T.$$
The reconstruction error of a data vector is defined as the squared distance to the original data. The average error is the average of the errors of all the considered data vectors:
$$Err_i = \frac{1}{n}\sum_{j=1}^{n}||Y_{*j}-Y^i_{*j}||^2$$ where $Y_{*j}$ denotes the $j$th column of $Y$. The principal components minimize the reconstruction error of this dimension reduction.

The average reconstruction errors for different numbers of different numbers of principal components are displayed in Figure~\ref{fig:reduce}.
It can be seen that the error goes down quickly during about the first 20 components but the differences become smaller after that.

Figure~\ref{fig:reduce} displays the result of doing dimension reduction and reconstruction for the 10 first digits using different numbers of principal components.
All the digits remain readable after reducing them to 16 first principal components.
With much fewer components, it becomes impossible to distinguish eg. 4 and 9 from each other.

The dimension reduction performed by projection to the principal component directions is also a form of lossy compression, because it represents approximately the same images using a smaller amount of values.
The method does not always reduce the total data size, as we need to store the first principal components to reproduce the data.
The size of the PCA matrix doesn't depend on the number of data vectors, only on the number of variables, so we can save space if we have a very large amount of similarly distributed data.
The data vectors should follow a similar distribution so that all of them can be recreated accurately using the same principal components.

\begin{figure}\centering
	\includegraphics[scale=0.5]{error}
	\includegraphics[scale=\sscale]{digitreduce}
	\caption{Left: the average reconstruction error of dimension reduction as a function of the number of principal components.
	Right: The first 10 digits after reducing them to 1, 2, 4, 8, 16, 32, 64 and 128 dimensional subspaces spanned by the first principal components.
The rightmost column contains the original digits.
The digits where preprocessed before the reduction and the preprocessing steps where inversed after the reduction.}\label{fig:reduce}
\end{figure}

\subsection{}

The dimension reduction with PCA can be used as a denoising method.
As we saw, digit images can be represented by a small number of principal component directions, so we can ignore all the variance in other directions as noise.
Since we use only a small number of principal components, the subspace spanned by their directions is only a small fraction of the whole space, so we ignore a large amounf of the noise.
Note that the principal components we use are the principal components of the non-noisy digit images, because we don't want the components to represent the noise but the actual signal parts of the data.

Figure~\ref{fig:denoise} displays the result of denoising the noisy images by projecting them to some of the first principal components. It can be seen that many images are much more readable than the original ones when reduction is done with 16 components. With 8 components some digits become even clearer, but many also become too blurry to recognize. With 32 components most digits are quite hard to read, as much of the noise still remains.

\begin{figure}\centering
\newcommand{\dns}{0.45}
\subfigure[Original data]{\includegraphics[scale=\dns]{noisy}\label{dno}}
\subfigure[Reduction using 8 components]{\includegraphics[scale=\dns]{denoise8}\label{dn8}}
\subfigure[Reduction using 16 components]{\includegraphics[scale=\dns]{denoise16}\label{dn16}}
\subfigure[Reduction using 32 components]{\includegraphics[scale=\dns]{denoise32}\label{dn32}}
%	\includegraphics[scale=\dns]{denoise64}
\caption{The result of denoising noisy digit images using dimension reduction. The images are denoised by projecting them to low-dimensional subspace spanned by the first principal components of non-noisy digit images. The image~\ref{dno} contains the original images and the images~\ref{dn8},\ref{dn16},\ref{dn32} display the result of reduction to 8, 16 and~32 components respectively.}\label{fig:denoise}
\end{figure}

\end{document}
