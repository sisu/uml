\documentclass{article}
\usepackage[utf8]{inputenc}
\usepackage{amssymb, amsmath,hyperref,verbatim,listings,graphicx,subfigure,fullpage}

\begin{document}

\title{UML computer project 2}
\author{
Juha-Antti Isojärvi\\
013455341 \\
Department of Mathematics and Statistics\\
Master student
\and
Mikko Sysikaski\\
013573016\\
Department of Computer Science\\
Master student}
\date{}
\maketitle

\section{Exercise set 1}
In the first set of exercise we compared results of applying ICA to
uniformly distributed and gaussian data.
\subsection{Exercise 1}
First we needed some data. We were given two mixing matrices 
\[
A_1 = \left[ \begin{array}{ccc}
0.4483 & -1.6730  \\
2.1907 & -1.4836  \end{array} \right], \quad
A_2 = \left[ \begin{array}{ccc}
0 & -1.7321  \\
1.7321 & -2.0  \end{array} \right].
\]
These matrices were then used to mix a sample \textbf{s} of uniformly distributed
data and a sample \textbf{n} of gaussian data, each containing 5000
points. The results of mixing are denoted 
\[
\begin{array}{ccc}
\textbf{x}_1 = A_1\textbf{s}, \\
\textbf{x}_2 = A_2\textbf{s}, \\
\textbf{y}_1 = A_1\textbf{n}, & \textup{and} \\
\textbf{y}_2 = A_2\textbf{n}.
\end{array}
\]

Scatter plots of the original and mixed data can be seen in figure
\ref{fig:scatterE21}. In our previous report we described how standard
normally distributed data with zero mean and covariance $\Sigma$ has a
multinormal distribution with zero mean and covariance $AA^T$, after
it is linearly transformed with a matrix $A$. Thus 
\[
E \textbf{y}_1 = 0 = E \textbf{y}_2,
\]
and 
\[
\textup{cov}(\textbf{y}_1) = A_1A_1^T = \left[ \begin{array}{ccc}
3.000 & 3.4642  \\
3.4642 & 7.000  \end{array} \right], \quad
\textup{cov}(\textbf{y}_1) = A_1A_1^T = \left[ \begin{array}{ccc}
3.000 & 3.4642  \\
3.4642 & 7.000  \end{array} \right].
\] 
Notice that though the mixing matrices are different, the covariance
of the mixed data is equal. Below the computed sample covariance and mean of the mixed data:
\begin{verbatim}
> mean(y1)
[1] -0.007008821
> mean(y2)
[1] -0.01386588
> cov(y1)
         [,1]     [,2]
[1,] 2.898259 3.317027
[2,] 3.317027 6.749532
> cov(y2)
         [,1]     [,2]
[1,] 2.917094 3.328705
[2,] 3.328705 6.733001
\end{verbatim}
As can be seen, the sample covariance and mean are roughly equal to the covariance and mean of the
underlying distribution, as is expected.

\begin{figure}\centering
	\includegraphics[trim = 0cm 5.9cm 0cm 0cm, clip = true,
totalheight=0.5\textheight]{scatterPlotE21.pdf}
	\caption{Scatter plots of the data before and after
          mixing. The first column shows the unmixed data, the second column
          the data mixed with $A_1$ and the third the data mixed with $A_2$.} \label{fig:scatterE21}
\end{figure}

\subsection{Exercise 2}
Now that we had our generated artificial data, the first step of ICA
is to whiten the data. The results of applying the PCA based whitening
transformation can be seen plotted in figure \ref{fig:scatterWhiteE22}.
The whitening matrices used were:
\begin{verbatim}
> whiteningY1
           PC1        PC2
[1,] 0.1695626  0.2945006
[2,] 0.8716748 -0.5018783
> whiteningY2
          PC1        PC2
[1,] 0.170350  0.2939907
[2,] 0.870344 -0.5043123
> whiteningX1
           PC1        PC2
[1,] 0.1639257  0.2850347
[2,] 0.8680057 -0.4991970
> whiteningX2
            PC1        PC2
[1,] -0.1635482 -0.2860464
[2,] -0.8673773  0.4959264
\end{verbatim}
We failed to see the connection between the whitening matrices and the
mixing matrices.
% PCA:
% \begin{verbatim}
% > prcomp(y1)
% Standard deviations:
% [1] 2.9426781 0.9942015

% Rotation:
%            PC1        PC2
% [1,] 0.4989681  0.8666204
% [2,] 0.8666204 -0.4989681
% > prcomp(y2)
% Standard deviations:
% [1] 2.9430913 0.9941372

% Rotation:
%            PC1        PC2
% [1,] 0.5013556  0.8652413
% [2,] 0.8652413 -0.5013556
% > prcomp(x1)
% Standard deviations:
% [1] 3.0412644 0.9986868

% Rotation:
%            PC1        PC2
% [1,] 0.4985415  0.8668658
% [2,] 0.8668658 -0.4985415
% > prcomp(x2)
% Standard deviations:
% [1] 3.034897 1.000858

% Rotation:
%             PC1        PC2
% [1,] -0.4963519 -0.8681214
% [2,] -0.8681214  0.4963519
%\end{verbatim}
\begin{figure}\centering
	\includegraphics[totalheight=0.5\textheight]{scatterPlotOfWhitenedE22.pdf}
	\caption{Scatter plots of the whitened data.} \label{fig:scatterWhiteE22}
\end{figure}

\subsection{Exercise 3}
ICA can be solved by maximizing kurtosis. In figure
\ref{fig:kurtosisAlphaE23} the kurtosis of each mixed whitened
data, projected onto a unit vector \textbf{w} which forms an angle
$\alpha \in [0,\pi]$ with the $x$-axis, is plotted as a function of
alpha. 

The plots show larger absolute values of kurtosis for the uniform data, as for
the gaussian data. Also, the kurtosis of the uniform distribution is a
sine curve, which is fairly regular as should be expected for
nongaussian ICs. Coincidentally, the gaussian distribution also show a
sine-like curve for kurtosis. In repeated further tests the kurtosis
of the gaussian data showed more irregularity. In fact the gaussian
data should have kurtosis of zero, and different values of kurtosis
are the result of chance.

Kurtosis was maximized for the following $\alpha$:
\begin{verbatim}
> maxAlphaY1
[1] 1.006316
> maxAlphaY2
[1] 0.7390133
> maxAlphaX1
[1] 0.7798949
> maxAlphaX2
[1] 0.5220264
\end{verbatim}
\begin{figure}\centering
	\includegraphics[totalheight=0.5\textheight]{kurtosisAlphaE23.pdf}
	\caption{Kurtosis as a function of alpha.} \label{fig:kurtosisAlphaE23}
\end{figure}
\subsection{Exercise 4}
The optimal projection vectors $\textbf{b}_i$ are the rows of the inverse of the
whitened matrix $A_w$
\[
B = \left[ \begin{array}{ccc}
b_1^T  \\
b_2 ^T \end{array} \right] = A_w^{-1}.
\]
$A_w$ is the result of transforming the original $A$ with the
whitening matrix denoted here by $W$:
\[
A_w = WA.
\]
Therefore $A$ can be estimated using the whitening matrix and the optimal
vectors in the following manner:
\[
A = W^-1B^-1.
\]
Our estimates are shown below:
\begin{verbatim}
> estA(x1,1000)
          [,1]     [,2]
[1,] 0.4213442 1.672819
[2,] 2.1730399 1.504597
> A1
       [,1]    [,2]
[1,] 0.4483 -1.6730
[2,] 2.1907 -1.4836
> estA(x2,1000)
            [,1]     [,2]
[1,] -0.02973494 1.726432
[2,]  1.70379490 2.014105
> A2
       [,1]    [,2]
[1,] 0.0000 -1.7321
[2,] 1.7321 -2.0000
> estA(y1,1000)
          [,1]       [,2]
[1,] -1.373120  1.0530153
[2,] -2.634125 -0.1839215
> A1
       [,1]    [,2]
[1,] 0.4483 -1.6730
[2,] 2.1907 -1.4836
> estA(y2,1000)
          [,1]     [,2]
[1,] -1.054454 1.379936
[2,] -2.586195 0.514233
> A2
       [,1]    [,2]
[1,] 0.0000 -1.7321
[2,] 1.7321 -2.0000
\end{verbatim}
Notice that the uniform data gives good estimates for the mixing
matrices, whereas the gaussian data gives bad estimates.
\subsection{Exercise 5}
Exercise 5 was to explain the above result: why is it possible to find
an estimate for $A_1$ and $A_2$ for the uniform data, but not for the
gaussian data?

The reason for this is that the whitened matrix $A_w$ is orthogonal,
and therefore, if \textbf{s} is gaussian, $A_ws$ is also gaussian and
white. Now there is no unique solution to the optimal projection
$b_i^TA_ws$, because of orthogonality assumption for the estimated ICs
$b_i$.

\section{Exercise set 2}
Next we were given the task of performing ICA on 32 independent
variables following a Laplacian distribution.
\subsection{Exercise 1}
First we needed to artificially generate the data. This was done using
standard R function rlaplace. Then we transformed the data by doing
the following:

\subsection{Exercise 2}
\subsection{Exercise 3}
\begin{verbatim}
> ICASymmetric(whitenedX1, 2)
          [,1]       [,2]
[1,] 0.6973196 -0.7167604
[2,] 0.7167604  0.6973196
\end{verbatim}
\subsection{Exercise 4}
The matrix transforming $\textbf{s} \to \tilde{\textbf{y}}$, $y_m = \sum_{i=1}^m s_i$, is given by
the lower triangular matrix of which every nonzero element is equal to
one. An example in three dimensions:
\[
\Sigma =
\left[ \begin{array}{ccc}
1 & 0 & 0 \\
1 & 1 & 0 \\
1 & 1 & 1 \end{array} \right],
\quad
\tilde{\textbf{y}} = \Sigma \textbf{s} = (s_1, s_1 + s_2, s_1 + s_2 + s_3)^T.
\]
To normalize $y_m$, the variance is required. After estimating
the variance, the transformation $\tilde{\textbf{y}} \to \textbf{y}$
is achieved via multiplication by the diagonal matrix $D$ containing the
standard deviations: 
\[
\textbf{y} = D \tilde{\textbf{y}}= \textup{diag} \left( \frac{1}{\sqrt{\textup{var}(y_i)}}
\right) \tilde{\textbf{y}}.
\]

Now, after transforming $\textbf{s} \to \tilde{\textbf{y}}$ and
estimating the variance, one gets matrices $D$ and $\Sigma$. Their
product $A = D \Sigma$ gives the transformation matrix $\textbf{s} \to
\textbf{y}$:
\[
\textbf{y} = A\textbf{s}.
\]
\subsection{Exercise 5}
10000 points:
\begin{verbatim}
> mean(1/32^2 * (Ahat-A)^2)
[1] 6.090882e-05
\end{verbatim}
20000 points:
\begin{verbatim}
> mean(1/32^2 * (Ahat-A)^2)
[1] 5.634683e-05
\end{verbatim}


\section{Exercise set 3}

\subsection{Exercise 1}

In the third exercise we were given six images which were weighted sums of six hidden source images, and the goal was to recover the original images from the mixtures.
The visualization of the mixed images is shown in Figure~\ref{fig:mixed}.

For a single image, we can assume that each pixel is a sample of a random variable with distribution defined by the color distribution of the image.
In order to apply the ICA model we have to assume that the random variables are independent of each other.
Since the source images are unrelated to each other the pixels of different source images at the same position are not directly correlated.
However, since the pixel values are not truly random, there are usually some hidden depencies between real world images.
For instance, suppose that $s_1$ and $s_2$ are two source images and $p$ and $q$ are pixel positions.
If $(s_1)_p\approx(s_1)_q$, then $p$ and $q$ are more likely to be close to each other in the images than if the values are very different, and thus $(s_2)_p$ and $(s_2)_q$ are also likely to be quite close to each other.
In practice this correlation appears to be small enough that we can still separate a small number of images quite well.

\newcommand\iscale{0.45}
\begin{figure}\centering
	\includegraphics[scale=\iscale]{mixed}
	\caption{Mixed images used as a source of ICA.}\label{fig:mixed}
\end{figure}

\subsection{Exercise 2}

\begin{figure}\centering
	\includegraphics[scale=\iscale]{white}
	\caption{Result of doing a whitening transform for the images of Figure~\ref{fig:mixed}.}\label{fig:white}
\end{figure}

\subsection{Exercise 3}
\subsection{Exercise 4}

\begin{figure}\centering
	\includegraphics[scale=\iscale]{icares}
	\caption{Result of doing ICA on the whitened mixture of images.}\label{fig:icares}
\end{figure}


\end{document}
