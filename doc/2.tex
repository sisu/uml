\documentclass{article}
\usepackage[utf8]{inputenc}
\usepackage{amssymb, amsmath,hyperref,verbatim,listings,graphicx,subfigure,fullpage}

\begin{document}

\title{UML computer project 2}
\author{
Juha-Antti Isojärvi\\
013455341 \\
Department of Mathematics and Statistics\\
Master student
\and
Mikko Sysikaski\\
013573016\\
Department of Computer Science\\
Master student}
\date{}
\maketitle

\section{Exercise set 1}
\subsection{Exercise 1}
Sample covariance and mean:
\begin{verbatim}
> mean(y1)
[1] -0.007008821
> mean(y2)
[1] -0.01386588
> cov(y1)
         [,1]     [,2]
[1,] 2.898259 3.317027
[2,] 3.317027 6.749532
> cov(y2)
         [,1]     [,2]
[1,] 2.917094 3.328705
[2,] 3.328705 6.733001
\end{verbatim}
Distributions are multinormal with zero mean and covariance $A_1A_1^T$
and $A_2A_2^T$. 

\begin{figure}\centering
	\includegraphics{scatterPlotE21.pdf}
	\caption{Scatter plots of the data.} \label{fig:scatterE21}
\end{figure}

\subsection{Exercise 2}
The whitening matrices were:
\begin{verbatim}
> whiteningY1
           PC1        PC2
[1,] 0.1695626  0.2945006
[2,] 0.8716748 -0.5018783
> whiteningY2
          PC1        PC2
[1,] 0.170350  0.2939907
[2,] 0.870344 -0.5043123
> whiteningX1
           PC1        PC2
[1,] 0.1639257  0.2850347
[2,] 0.8680057 -0.4991970
> whiteningX2
            PC1        PC2
[1,] -0.1635482 -0.2860464
[2,] -0.8673773  0.4959264
\end{verbatim}
PCA:
\begin{verbatim}
> prcomp(y1)
Standard deviations:
[1] 2.9426781 0.9942015

Rotation:
           PC1        PC2
[1,] 0.4989681  0.8666204
[2,] 0.8666204 -0.4989681
> prcomp(y2)
Standard deviations:
[1] 2.9430913 0.9941372

Rotation:
           PC1        PC2
[1,] 0.5013556  0.8652413
[2,] 0.8652413 -0.5013556
> prcomp(x1)
Standard deviations:
[1] 3.0412644 0.9986868

Rotation:
           PC1        PC2
[1,] 0.4985415  0.8668658
[2,] 0.8668658 -0.4985415
> prcomp(x2)
Standard deviations:
[1] 3.034897 1.000858

Rotation:
            PC1        PC2
[1,] -0.4963519 -0.8681214
[2,] -0.8681214  0.4963519
\end{verbatim}
\begin{figure}\centering
	\includegraphics{scatterPlotOfWhitenedE22.pdf}
	\caption{Scatter plots of the whitened data.} \label{fig:scatterWhiteE22}
\end{figure}

\subsection{Exercise 3}
Kurtosis maximized for these alpha:
\begin{verbatim}
> maxAlphaY1
[1] 1.006316
> maxAlphaY2
[1] 0.7390133
> maxAlphaX1
[1] 0.7798949
> maxAlphaX2
[1] 0.5220264
\end{verbatim}
\begin{figure}\centering
	\includegraphics{kurtosisAlphaE23.pdf}
	\caption{Kurtosis as a function of alpha.} \label{fig:kurtosisAlphaE23}
\end{figure}
\subsection{Exercise 4}
\begin{verbatim}
> estA(x1,1000)
          [,1]     [,2]
[1,] 0.4213442 1.672819
[2,] 2.1730399 1.504597
> A1
       [,1]    [,2]
[1,] 0.4483 -1.6730
[2,] 2.1907 -1.4836
> estA(x2,1000)
            [,1]     [,2]
[1,] -0.02973494 1.726432
[2,]  1.70379490 2.014105
> A2
       [,1]    [,2]
[1,] 0.0000 -1.7321
[2,] 1.7321 -2.0000
> estA(y1,1000)
          [,1]       [,2]
[1,] -1.373120  1.0530153
[2,] -2.634125 -0.1839215
> A1
       [,1]    [,2]
[1,] 0.4483 -1.6730
[2,] 2.1907 -1.4836
> estA(y2,1000)
          [,1]     [,2]
[1,] -1.054454 1.379936
[2,] -2.586195 0.514233
> A2
       [,1]    [,2]
[1,] 0.0000 -1.7321
[2,] 1.7321 -2.0000
\end{verbatim}
\subsection{Exercise 5}

\section{Exercise set 2}

\subsection{Exercise 1}
\subsection{Exercise 2}
\subsection{Exercise 3}
\begin{verbatim}
> ICASymmetric(whitenedX1, 2)
          [,1]       [,2]
[1,] 0.6973196 -0.7167604
[2,] 0.7167604  0.6973196
\end{verbatim}
\subsection{Exercise 4}
The matrix transforming $\textbf{s} \to \tilde{\textbf{y}}$, $y_m = \sum_{i=1}^m s_i$, is given by
the lower triangular matrix of which every nonzero element is equal to
one. An example in three dimensions:
\[
\Sigma =
\left[ \begin{array}{ccc}
1 & 0 & 0 \\
1 & 1 & 0 \\
1 & 1 & 1 \end{array} \right],
\quad
\tilde{\textbf{y}} = \Sigma \textbf{s} = (s_1, s_1 + s_2, s_1 + s_2 + s_3)^T.
\]
To normalize $y_m$, the variance is required. After estimating
the variance, the transformation $\tilde{\textbf{y}} \to \textbf{y}$
is achieved via multiplication by the diagonal matrix $D$ containing the
standard deviations: 
\[
\textbf{y} = D \tilde{\textbf{y}}= \textup{diag} \left( \frac{1}{\sqrt{\textup{var}(y_i)}}
\right) \tilde{\textbf{y}}.
\]

Now, after transforming $\textbf{s} \to \tilde{\textbf{y}}$ and
estimating the variance, one gets matrices $D$ and $\Sigma$. Their
product $A = D \Sigma$ gives the transformation matrix $\textbf{s} \to
\textbf{y}$:
\[
\textbf{y} = A\textbf{s}.
\]
\subsection{Exercise 5}



\end{document}
